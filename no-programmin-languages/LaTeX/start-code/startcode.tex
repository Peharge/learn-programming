\documentclass{article}
\usepackage[utf8]{inputenc}
\usepackage{amsmath}  % für mathematische Ausdrücke

\title{Willkommen zu meinem LaTeX-Dokument}
\author{Dein Name}
\date{\today}

\begin{document}

\maketitle  % Titel, Autor und Datum einfügen

\section{Einleitung}
Dies ist ein einfaches Beispiel für ein LaTeX-Dokument. Hier kannst du beginnen, dein Dokument zu schreiben. LaTeX ist besonders nützlich für wissenschaftliche Arbeiten und mathematische Formeln.

\section{Mathematik}
LaTeX ermöglicht es, mathematische Ausdrücke sehr elegant zu schreiben. Hier ist ein Beispiel für eine einfache Formel:

\[
E = mc^2
\]

\subsection{Unterabschnitt}
Du kannst auch Unterabschnitte erstellen, um dein Dokument weiter zu strukturieren.

\section{Fazit}
LaTeX ist ein leistungsstarkes Werkzeug, um wissenschaftliche und technische Dokumente zu erstellen. Mit LaTeX kannst du klare und strukturierte Dokumente erstellen.

\end{document}
